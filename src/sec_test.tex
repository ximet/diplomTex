\section{Тестирование программного средства}
\label{sec:test}

Для обеспечения качества программного средства используются модульные и интеграционные тесты. Модульные тесты позволяют проверять отдельные участки программного средства на правильность их выполнения при определенных входных параметрах. Это позволяет достаточно быстро проверить, не привело ли очередное изменение кода к появлению определенного рода ошибок в уже оттестированных местах приложения, а также облегчает обнаружения и устранения таких ошибок. Цель модульного тестирования - изолировать отдельные участки приложения и доказать, что по отдельности эти части являются работоспособными. 

Интеграционные тесты позволяют проверить правильность взаимодействия различных модулей программного средства друг с другом. При интеграционном тестирование отдельные модули приложения объеденяются в специальные модули и тестируются в совокупности.

В таблице 5.1 представлены тестовые сценарии, разработанные в ходе выполнения дипломного проекта для данного мобильного программного средства.


  \begin{longtable}{| >{\raggedright}p{0.20\linewidth} 
                  | >{\raggedright}p{0.28\linewidth} 
                  | >{\raggedright}p{0.28\linewidth} 
                  | >{\raggedright\arraybackslash}p{0.12\linewidth}|}
   \caption{Тестовые сценарии} \label{tab:long} \\

   \hline
   Идентификатор тест-кейса & Описание тест-кейса & Ожидаемый результат & Тестовый сценарий пройден Да/Нет \\
   \endfirsthead

\multicolumn{3}{l}%
{{\raggedright Продолжение таблицы \thetable{}}} \\
\hline
   Идентификатор тест-кейса & Описание тест-кейса & Ожидаемый результат & Тестовый сценарий пройден Да/Нет \\
\endhead
   \hline
   Login-1 &
   			\vspace{-6.5mm} 
   			\begin{enumerate} 
   				\item[1)] Ввести в поле <<Login>> и поле <<Passcode>> не корректные значения;
				\item[2)] нажать на кнопку <<Log In>>.
			\end{enumerate}
   			 & Появится сообщение <<Invalid login>> & Да \\

   \hline
   Login-2 & 
   			\vspace{-6.5mm}
   			\begin{enumerate} 
   				\item[1)] Ввести в поле корректное значение <<Login>>;
				\item[2)] ввести в поле <<Passcode>> значение из символов;
				\item[3)] нажать на кнопку <<Log In>>.
			\end{enumerate}
			& Появится сообщение <<Passcode length should be 4 number>> & Да \\

   \hline
   Login-3 & 
   			\vspace{-6.5mm}
   			\begin{enumerate} \item[1)] Ввести в поле корректное значение <<Login>>;
				\item[2)] ввести в поле <<Passcode>> значение из более чем 4 цифр;
				\item[3)] нажать на кнопку <<Log In>>.
			\end{enumerate}
   			& Появится сообщение <<Passcode length should be 4 number>> & Да \\
   	\hline
   Login-4 & 
   			\vspace{-6.5mm}
   			\begin{enumerate} \item[1)] Ввести в поле корректное значение <<Login>>;
				\item[2)] ввести в поле <<Passcode>> корректное значение из 4 цифр;
				\item[3)] нажать на кнопку <<Log In>>.
			\end{enumerate}
   			& 
   			\vspace{-6.5mm}
   			\begin{enumerate} \item[1)] Закрывается страница авторизации;
				\item[2)] появляется главная страница приложения.
			\end{enumerate}
			& Да \\
   \hline
   Login-5 & \vspace{-6.5mm} \begin{enumerate} \item[1)] Ввести в поле корректное значение <<Login>>;
				\item[2)] ввести в поле <<Passcode>> значение из 4 цифр не являющуюся паролем данного пользователя;
				\item[3)] нажать на кнопку <<Log In>>.
			\end{enumerate}
   			& Появится сообщение <<Invalid Passcode>> & Да \\
   	\hline
   	Login-6 & \vspace{-6.5mm} \begin{enumerate} \item[1)] Ввести в поле несколько пробелов, а после корректное значение <<Login>>;
				\item[2)] ввести в поле <<Passcode>> корректное значение из 4 цифр;
				\item[3)] нажать на кнопку <<Log In>>.
			\end{enumerate}
   			& \vspace{-6.5mm} \begin{enumerate} \item[1)] Закрывается страница авторизации;
				\item[2)] появляется главная страница приложения.
			\end{enumerate}
			& Да \\
   	\hline
   	Register-1 & \vspace{-6.5mm} \begin{enumerate} \item[1)] Ввести в поле значение <<Login>>;
				\item[2)] ввести в поле <<Passcode>> корректное значение из 4 цифр без пробелов;
				\item[3)] ввести в поле <<Confirm Passcode>> другое значение из 4 цифр без пробелов;
				\item[4)] нажать на кнопку <<Register>>.
			\end{enumerate}
   			& Появляется сообщение <<Not corrected passcode>>;
			& Да \\
   	\hline
   	Register-2 & \vspace{-6.5mm} \begin{enumerate} \item[1)] Ввести в поле значение <<Login>>;
				\item[2)] ввести в поле <<Passcode>> корректное значение из 4 цифр без пробелов;
				\item[3)] ввести в поле <<Confirm Passcode>> схожее значение с полем <<Passcode>>
				\item[4)] нажать на кнопку <<Register>>.
			\end{enumerate}
   			& \vspace{-6.5mm} \begin{enumerate} \item[1)] Закрывается страница авторизации;
				\item[2)] появляется страница заполнения профиля.
			\end{enumerate}
			& Да \\
   	\hline
   	Register-3 & \vspace{-6.5mm} \begin{enumerate} \item[1)] Ввести в поле значение <<Login>>;
				\item[2)] ввести в поле <<Passcode>> значение из более чем 4 цифр без пробелов;
				\item[3)] ввести в поле <<Confirm Passcode>> схожее значение с полем <<Passcode>>
				\item[4)] нажать на кнопку <<Register>>.
			\end{enumerate}
   			& Появится сообщение <<Passcode length should be 4 number>> & Да \\
   	\hline
   	Register-4 & \vspace{-6.5mm} \begin{enumerate} \item[1)] Ввести в поле значение <<Login>> существующего пользователя;
				\item[2)] ввести в поле <<Passcode>> корректное значение из 4 цифр без пробелов;
				\item[3)] ввести в поле <<Confirm Passcode>> схожее значение с полем <<Passcode>>
				\item[4)] нажать на кнопку <<Register>>.
			\end{enumerate}
   			& Появится сообщение <<User consist in system>> & Да \\
   	\hline
   	Register-5 & \vspace{-6.5mm} \begin{enumerate} \item[1)] Ввести в поле значение <<Login>>;
				\item[2)] ввести в поле <<Passcode>> строку с пробелами и следом 4 цифры;
				\item[3)] ввести в поле <<Confirm Passcode>> схожее значение с полем <<Passcode>>
				\item[4)] нажать на кнопку <<Register>>.
			\end{enumerate}
   			& Появится сообщение <<Passcode can't contain space>> & Да \\
   	\hline
   	Register-6 & \vspace{-6.5mm} \begin{enumerate} \item[1)] Ввести в поле значение <<Login>>;
				\item[2)] ввести в поле <<Passcode>> корректное значение из 4 цифр без пробелов;
				\item[3)] ввести в поле <<Confirm Passcode>> схожее значение из 4 цифр с пробелов;
				\item[4)] нажать на кнопку <<Register>>.
			\end{enumerate}
   			& Появится сообщение <<Passcode can't contain space>> & Да \\
   	\hline
   	Logout-1 & \vspace{-6.5mm} \begin{enumerate} \item[1)] Ввести в поле корректное значение <<Login>>;
				\item[2)] ввести в поле <<Passcode>> корректное значение из 4 цифр;
				\item[3)] нажать на кнопку <<Log In>>;
				\item[3)] нажать на кнопку <<Log Out>>.
			\end{enumerate}
   			& Появится страница с окном авторизации & Да \\
   \hline
  \end{longtable}