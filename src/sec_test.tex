\section{Тестирование программного средства}
\label{sec:test}

Для обеспечения качества программного средства используются модульные и интеграционные тесты. Модульные тесты позволяют проверять отдельные участки программного средства на правильность их выполнения при определенных входных параметрах. Это позволяет достаточно быстро проверить, не привело ли очередное изменение кода к появлению определенного рода ошибок в уже оттестированных местах приложения, а также облегчает обнаружения и устранения таких ошибок. Цель модульного тестирования - изолировать отдельные участки приложения и доказать, что по отдельности эти части являются работоспособными. 

Интеграционные тесты позволяют проверить правильность взаимодействия различных модулей программного средства друг с другом. При интеграционном тестирование отдельные модули приложения объеденяются в специальные модули и тестируются в совокупности.

В таблице 5.1 представлены тестовые сценарии, разработанные в ходе выполнения дипломного проекта для данного мобильного программного средства.


  \begin{longtable}{| >{\raggedright}p{0.20\linewidth} 
                  | >{\raggedright}p{0.28\linewidth} 
                  | >{\raggedright}p{0.28\linewidth} 
                  | >{\raggedright\arraybackslash}p{0.12\linewidth}|}
   \hline
   Идентификатор тест-кейса & Описание тест-кейса & Ожидаемый результат & Тестовый сценарий пройден Да/Нет \\
   \hline
   Login-1 & \begin{enumerate}
				\item[1)] Ввести в поле <<Login>> и поле <<Passcode>> не корректные значения;
				\item[2)] Нажать на кнопку <<Log In>>;
			\end{enumerate}
   			 & Появится сообщение <<Invalid login>> & Да \\

   \hline
   Login-2 & \begin{enumerate}
				\item[1)] Ввести в поле корректное значение <<Login>>;
				\item[2)] Ввести в поле <<Passcode>> значение из символов;
				\item[3)] Нажать на кнопку <<Log In>>;
			\end{enumerate}
			& Появится сообщение <<Passcode length should be 4 number>> & Да \\

   \hline
   Login-3 & \begin{enumerate}
				\item[1)] Ввести в поле корректное значение <<Login>>;
				\item[2)] Ввести в поле <<Passcode>> значение из более чем 4 цифр;
				\item[3)] Нажать на кнопку <<Log In>>;
			\end{enumerate}
   			& Появится сообщение <<Passcode length should be 4 number>> & Да \\
   	\hline
   Login-4 & \begin{enumerate}
				\item[1)] Ввести в поле корректное значение <<Login>>;
				\item[2)] Ввести в поле <<Passcode>> корректное значение из 4 цифр;
				\item[3)] Нажать на кнопку <<Log In>>;
			\end{enumerate}
   			& 
   			\begin{enumerate}
				\item[1)] Закрывается страница авторизации;
				\item[2)] Появляется главная страница приложения;
			\end{enumerate}
			& Да \\
   \hline
   Login-5 & \begin{enumerate}
				\item[1)] Ввести в поле корректное значение <<Login>>;
				\item[2)] Ввести в поле <<Passcode>> значение из 4 цифр не являющуюся паролем данного пользователя;
				\item[3)] Нажать на кнопку <<Log In>>;
			\end{enumerate}
   			& Появится сообщение <<Invalid Passcode>> & Да \\
   	\hline
   	Login-6 & \begin{enumerate}
				\item[1)] Ввести в поле несколько пробелов, а после корректное значение <<Login>>;
				\item[2)] Ввести в поле <<Passcode>> корректное значение из 4 цифр;
				\item[3)] Нажать на кнопку <<Log In>>;
			\end{enumerate}
   			&
   			\begin{enumerate}
				\item[1)] Закрывается страница авторизации;
				\item[2)] Появляется главная страница приложения;
			\end{enumerate}
			& Да \\
   	\hline
  \end{longtable}