\section{Тестирование программного средства}
\label{sec:testing}

Для тестирования программного средства используются модульные и интеграционные тесты. Модульные тесты позволяют проверять отдельные участки программного средства на правильность их выполнения при определенных входных параметрах. Это позволяет достаточно быстро проверить, не привело ли очередное изменение кода к появлению определенного рода ошибок в уже оттестированных местах приложения, а также облегчает обнаружения и устранения таких ошибок. Цель модульного тестирования -- изолировать отдельные участки приложения и доказать, что по отдельности эти части являются работоспособными. 

Интеграционные тесты позволяют проверить правильность взаимодействия различных модулей программного средства друг с другом. При интеграционном тестирование отдельные модули приложения объеденяются в специальные модули и тестируются в совокупности.

Тест-кейсы для проверки базовой функциональности программного средства представлены в таблицах~\ref{sec:testing:configuration_register},~\ref{sec:testing:configuration_login} и~\ref{sec:testing:data_storing}.

\begin{longtable}[l]{| >{\raggedright}p{0.3\textwidth}
                     | >{\raggedright}p{0.3\textwidth}
                     | >{\raggedright\arraybackslash}p{0.3\textwidth}|}
  \caption{Тестирование регистрации}
  \label{sec:testing:configuration_register} \tabularnewline

  \hline
       Название тест-кейса и его описание & Ожидаемый результат & Фактический результат результат \\
   \hline
   Регистрация\\ 
   1) Ввести в поле <<Login>> значение не существующего пользователя; \\
   2) ввести в поле <<Passcode>> значение из 4 цифр без пробелов; \\  
   3) ввести в поле <<Confirm passcode>> схожее значение с полем <<Passcode>>; \\ 
   4) нажать на кнопку <<Register>>.

   &
   1) Закрывается страница авторизации; \\
   2) появляется страница заполнения профиля.

   & Тест пройден \\
   \hline
\end{longtable}

\begin{longtable}[l]{| >{\raggedright}p{0.3\textwidth}
                     | >{\raggedright}p{0.3\textwidth}
                     | >{\raggedright\arraybackslash}p{0.3\textwidth}|}
  \caption{Тестирование авторизации}
  \label{sec:testing:configuration_login} \tabularnewline

  \hline
       Название тест-кейса и его описание & Ожидаемый результат & Фактический результат результат \\
   \hline
   Авторизация\\ 
   1) Ввести в поле <<Login>> значение существующего пользователя; \\
   2) ввести в поле <<Passcode>> значение из 4 цифр без пробелов, являющееся паролем пользователя; \\  
   4) нажать на кнопку <<Login>>.

   &
   1) Закрывается страница авторизации; \\
   2) появляется главная страница приложения.

   & Тест пройден \\
   \hline
\end{longtable}



\begin{longtable}[l]{| >{\raggedright}p{0.3\textwidth}                     
                     | >{\raggedright}p{0.3\textwidth}
                     | >{\raggedright\arraybackslash}p{0.3\textwidth}|}
  \caption{Тестирование выхода из спящего режима}
  \label{sec:testing:data_storing} \tabularnewline

  \hline
      Название тест-кейса и его описание & Ожидаемый результат & Фактический результат результат \\
   \hline
   Хранение данных в памяти\\ 
   1) Свернуть приложение и перейти на главную страницу мобильного телефона; \\
   2) запустить цикл удаления из памяти приложений;\\
   3) не очищать кэш;\\
   4) запустить приложение;\\
   5) набрать корректный, для текущего пользователя пасскод.

   &
   1) Закрывается страница пасскода;\\
   2) появляется главная страница приложения.\\

   &
   Тест пройден \\
   \hline
\end{longtable}

Тестирование основной функциональности мобильного программного средства будет представлен в таблицах~\ref{sec:testing:degradation_1},~\ref{sec:testing:degradation_2} и~\ref{sec:testing:degradation_3}.

\begin{longtable}[p]{| >{\raggedright}p{0.3\textwidth}                     
                     | >{\raggedright}p{0.3\textwidth}
                     | >{\raggedright\arraybackslash}p{0.3\textwidth}|}
  \caption{Тестирование процесса создания медицинского заключения}
  \label{sec:testing:degradation_1} \tabularnewline

  \hline
      Название тест-кейса и его описание & Ожидаемый результат & Фактический результат результат \\
   \hline
   Создание медицинского заключения\\ 
   1) На главной странице приложения открыть пункт <<Medical Card>>; \\
   2) нажать на кнопку <<+>>;\\
   3) заполнить все информационные поля в открывшейся странице;\\
   4) нажать на кнопку <<Save>>.

   &
   1) Закроется страница создания нового медицинского заключения;\\
   2) откроется главная страница приложения.

   &
   Тест пройден \\
   \hline
\end{longtable}

Создавать медицинские заключения можно и без использовании всех информативных полей. Главными полями медицинского заключения является: информация о докторе, дата создания медицинского заключения, общая информация о заключении.

Все остальные поля являются лишь дополнением, то есть информация об оплате и медикаментах, можно спокойно опустить и тест-кейс представленный в таблице~\ref{sec:testing:degradation_1} все равно будет пройден.

Стоит также упомянуть, что создание медицинского заключения является основной информационной базой мобильного приложения. Для всех пользователей приложения будет создана отдельная информационная база, тем самым можно быть уверенным, что данные будут конфиденциально сохранены.

\pagebreak

\begin{longtable}[p]{| >{\raggedright}p{0.3\textwidth}                     
                     | >{\raggedright}p{0.3\textwidth}
                     | >{\raggedright\arraybackslash}p{0.3\textwidth}|}
  \caption{Тестирование процесса редактирования медицинского заключения}
  \label{sec:testing:degradation_2} \tabularnewline

  \hline
      Название тест-кейса и его описание & Ожидаемый результат & Фактический результат результат \\
   \hline
   Редактирование медицинского заключения\\ 
   1) На главной странице приложения открыть пункт <<Medical Card>>; \\
   2) из представленного списка выбрать медицинское заключение;\\
   3) в открывшемся окне информации о медицинском заключении нажать на кнопку <<E>>;\\
   4) заполнить все информационные поля в открывшейся странице;\\
   5) нажать на кнопку <<Save>>.

   &
   1) Закроется страница редактирования медицинского заключения;\\
   2) откроется главная страница приложения.

   &
   Тест пройден \\
   \hline
\end{longtable}

\pagebreak

\begin{longtable}[p]{| >{\raggedright}p{0.3\textwidth}                     
                     | >{\raggedright}p{0.3\textwidth}
                     | >{\raggedright\arraybackslash}p{0.3\textwidth}|}
  \caption{Тестирование процесса удаления медицинского заключения}
  \label{sec:testing:degradation_3} \tabularnewline

  \hline
      Название тест-кейса и его описание & Ожидаемый результат & Фактический результат результат \\
   \hline
   Удаление медицинского заключения\\ 
   1) На главной странице приложения открыть пункт <<Medical Card>>; \\
   2) из представленного списка выбрать медицинское заключение;\\
   3) в открывшемся окне информации о медицинском заключении нажать на кнопку <<D>>.

   &
   1) Закроется страница информации медицинского заключения;\\
   2) откроется главная страница приложения.

   &
   Тест пройден \\
   \hline
\end{longtable}

Следует протестировать возможность системы анализировать по фотографии медицинское заключение и занесение его в список заключений представленный в таблице~\ref{sec:testing:regeneration}. 

Тестирование в основном производится стороннего анализатора Tesseract и как приложение взаимодействует с данным программным модулем. 

Также стоит упомянуть, что в данный момент, производится анализ медицинских заключений в определенном формате, то есть если воспользоваться другим форматом медицинского заключения, то данный тест будет не пройден.

\pagebreak

\begin{longtable}[p]{| >{\raggedright}p{0.3\textwidth}                     
                     | >{\raggedright}p{0.3\textwidth}
                     | >{\raggedright\arraybackslash}p{0.3\textwidth}|}
  \caption{Тестирование процесса анализа фотографии и занесение информации в новое заключение}
  \label{sec:testing:regeneration} \tabularnewline

  \hline
      Название тест-кейса и его описание & Ожидаемый результат & Фактический результат результат \\
   \hline
   Анализатор фотографии\\ 
   1) На главной странице приложения нажать на кнопку <<Photo Import>>; \\
   2) в открывшемся попапе запроса доступа к фотоаппарату мобильного телефона нажать на <<Открыть доступ>>; \\
   3) сделать фотографию и подтвердить ее; \\
   4) в открывшейся странице проверить целостность изображения; \\
   5) нажать на кнопку <<Import>>; \\
   6) дождаться окончания работы спинера; \\
   7) дозаполнить медицинское заключение в открывшемся окне; \\
   8) нажать на кнопку <<Save>>.

   &
   1) Закроется страница создания медицинского заключения;\\
   2) откроется главная страница приложения.

   &
   Тест пройден \\
   \hline
\end{longtable}

Необходимо проверить работоспособность экспорта данных из мобильного приложения, данный тест-кейс представлен в таблице~\ref{sec:testing:march_testing}.

\begin{longtable}[p]{| >{\raggedright}p{0.3\textwidth}                     
                     | >{\raggedright}p{0.3\textwidth}
                     | >{\raggedright\arraybackslash}p{0.3\textwidth}|}
  \caption{Тестирование экспорта данных}
  \label{sec:testing:march_testing} \tabularnewline

  \hline
      Название тест-кейса и его описание & Ожидаемый результат & Фактический результат результат \\
   \hline
   Экспорт данных\\ 
   1) На главной странице нажать на кнопку <<Convert to PDF>>; \\
   2) нажать на кнопку <<Export>>.

   &
   1) Закроется страница с списком медицинских заключений;\\
   2) откроется главная страница приложения.

   &
   Тест пройден \\
   \hline
\end{longtable}