\section{Тестирование программного средства}
\label{sec:testing}

Для тестирования программного средства используются модульные и интеграционные тесты. Модульные тесты позволяют проверять отдельные участки программного средства на правильность их выполнения при определенных входных параметрах. Это позволяет достаточно быстро проверить, не привело ли очередное изменение кода к появлению определенного рода ошибок в уже оттестированных местах приложения, а также облегчает обнаружения и устранения таких ошибок. Цель модульного тестирования -- изолировать отдельные участки приложения и доказать, что по отдельности эти части являются работоспособными. 

Интеграционные тесты позволяют проверить правильность взаимодействия различных модулей программного средства друг с другом. При интеграционном тестирование отдельные модули приложения объеденяются в специальные модули и тестируются в совокупности.

Тест-кейсы для проверки базовой функциональности программного средства представлены в таблицах \ref{sec:testing:configuration_register}, \ref{sec:testing:configuration_login}  и \ref{sec:testing:data_storing}.

\begin{longtable}[l]{| >{\raggedright}p{0.3\textwidth}
                     | >{\raggedright}p{0.3\textwidth}
                     | >{\raggedright\arraybackslash}p{0.3\textwidth}|}
  \caption{Тестирование регистрации}
  \label{sec:testing:configuration_register} \tabularnewline

  \hline
       Название тест-кейса и его описание & Ожидаемый результат & Фактический результат результат \\
   \hline
   Регистрация\\ 
   1) Ввести в поле <<Login>> значение не существующего пользователя; \\
   2) ввести в поле <<Passcode>> значение из 4 цифр без пробелов; \\  
   3) ввести в поле <<Confirm passcode>> схожее значение с полем <<Passcode>>; \\ 
   4) нажать на кнопку <<Register>>.

   &
   1) Закрывается страница авторизации; \\
   2) появляется страница заполнения профиля.

   & Тест пройден \\
   \hline
\end{longtable}

\begin{longtable}[l]{| >{\raggedright}p{0.3\textwidth}
                     | >{\raggedright}p{0.3\textwidth}
                     | >{\raggedright\arraybackslash}p{0.3\textwidth}|}
  \caption{Тестирование авторизации}
  \label{sec:testing:configuration_login} \tabularnewline

  \hline
       Название тест-кейса и его описание & Ожидаемый результат & Фактический результат результат \\
   \hline
   Авторизация\\ 
   1) Ввести в поле <<Login>> значение существующего пользователя; \\
   2) ввести в поле <<Passcode>> значение из 4 цифр без пробелов, являющееся паролем пользователя; \\  
   4) нажать на кнопку <<Login>>.

   &
   1) Закрывается страница авторизации; \\
   2) появляется главная страница приложения.

   & Тест пройден \\
   \hline
\end{longtable}



\begin{longtable}[l]{| >{\raggedright}p{0.3\textwidth}                     
                     | >{\raggedright}p{0.3\textwidth}
                     | >{\raggedright\arraybackslash}p{0.3\textwidth}|}
  \caption{Тестирование выхода из спящего режима}
  \label{sec:testing:data_storing} \tabularnewline

  \hline
      Название тест-кейса и его описание & Ожидаемый результат & Фактический результат результат \\
   \hline
   Хранение данных в памяти\\ 
   1) Свернуть приложение и перейти на главную страницу мобильного телефона; \\
   2) запустить цикл удаления из памяти приложений;\\
   3) не очищать кэш;\\
   4) запустить приложение;\\
   5) набрать корректный, для текущего пользователя пасскод.

   &
   1) Закрывается страница пасскода;\\
   2) появляется главная страница приложения.\\

   &
   Тест пройден \\
   \hline
\end{longtable}

Тестирование основной функциональности мобильного программного средства будет представлен в таблицах \ref{sec:testing:degradation_1}, \ref{sec:testing:degradation_2}  и \ref{sec:testing:degradation_3}.

\begin{longtable}[p]{| >{\raggedright}p{0.3\textwidth}                     
                     | >{\raggedright}p{0.3\textwidth}
                     | >{\raggedright\arraybackslash}p{0.3\textwidth}|}
  \caption{Тестирование процесса создания медицинского заключения}
  \label{sec:testing:degradation_1} \tabularnewline

  \hline
      Название тест-кейса и его описание & Ожидаемый результат & Фактический результат результат \\
   \hline
   Создание медицинского заключения\\ 
   1) На главной странице приложения открыть пункт <<Medical Card>>; \\
   2) нажать на кнопку <<+>>;\\
   3) заполнить все информационные поля в открывшейся странице;\\
   4) нажать на кнопку <<Save>>.

   &
   1) Закроется страница создания нового медицинского заключения;\\
   2) откроется главная страница приложения.

   &
   Тест пройден \\
   \hline
\end{longtable}

Создавать медицинские заключения можно и без использовании всех информативных полей. Главными полями медицинского заключения является: информация о докторе, дата создания медицинского заключения, общая информация о заключении.

Все остальные поля являются лишь дополнением, то есть информация об оплате и медикаментах, можно спокойно опустить и тест-кейс представленный в таблице \ref{sec:testing:degradation_1} все равно будет пройден.

Стоит также упомянуть, что создание медицинского заключения является основной информационной базой мобильного приложения. Для всех пользователей приложения будет создана отдельная информационная база, тем самым можно быть уверенным, что данные будут конфиденциально сохранены.

\pagebreak

\begin{longtable}[p]{| >{\raggedright}p{0.3\textwidth}                     
                     | >{\raggedright}p{0.3\textwidth}
                     | >{\raggedright\arraybackslash}p{0.3\textwidth}|}
  \caption{Тестирование процесса редактирования медицинского заключения}
  \label{sec:testing:degradation_2} \tabularnewline

  \hline
      Название тест-кейса и его описание & Ожидаемый результат & Фактический результат результат \\
   \hline
   Редактирование медицинского заключения\\ 
   1) На главной странице приложения открыть пункт <<Medical Card>>; \\
   2) из представленного списка выбрать медицинское заключение;\\
   3) в открывшемся окне информации о медицинском заключении нажать на кнопку <<E>>;\\
   4) заполнить все информационные поля в открывшейся странице;\\
   5) нажать на кнопку <<Save>>.

   &
   1) Закроется страница редактирования медицинского заключения;\\
   2) откроется главная страница приложения.

   &
   Тест пройден \\
   \hline
\end{longtable}

\pagebreak

\begin{longtable}[p]{| >{\raggedright}p{0.3\textwidth}                     
                     | >{\raggedright}p{0.3\textwidth}
                     | >{\raggedright\arraybackslash}p{0.3\textwidth}|}
  \caption{Тестирование процесса удаления медицинского заключения}
  \label{sec:testing:degradation_3} \tabularnewline

  \hline
      Название тест-кейса и его описание & Ожидаемый результат & Фактический результат результат \\
   \hline
   Удаление медицинского заключения\\ 
   1) На главной странице приложения открыть пункт <<Medical Card>>; \\
   2) из представленного списка выбрать медицинское заключение;\\
   3) в открывшемся окне информации о медицинском заключении нажать на кнопку <<D>>.

   &
   1) Закроется страница информации медицинского заключения;\\
   2) откроется главная страница приложения.

   &
   Тест пройден \\
   \hline
\end{longtable}

Следует протестировать поведение системы при наличии в памяти многократных ошибок. Для этого запускаются тесты из таблицы \ref{sec:testing:regeneration}.
\begin{longtable}[p]{| >{\raggedright}p{0.3\textwidth}                     
                     | >{\raggedright}p{0.3\textwidth}
                     | >{\raggedright\arraybackslash}p{0.3\textwidth}|}
  \caption{Тестирование процесса регенерации  памяти при сдвинутом периоде обновления}
  \label{sec:testing:regeneration} \tabularnewline

  \hline
      Название тест-кейса и его описание & Ожидаемый результат & Фактический результат результат \\
   \hline
   Регенерация памяти\\ 
   1) Инициализировать объект MemorySystem входными параметрами: путь к ini-файлу, путь к файлу конфигурации памяти, объем памяти, количество циклов, на которое сдвинут период обновления памяти; \\
   2) запустить цикл работы симулятора.

   &
   1) В консоли отображается информация о запуске системы и объеме симулируемой памяти;\\
   2) после наступления периода обновления появляется сообщение о несовпадении эталонной и тестовой сигнатур;\\
   3) появляется уведомление о запуске маршевого теста;\\
   4) маршевый тест пройден, неисправности не обнаружены.

   &
   Тест пройден \\
   \hline
\end{longtable}

В открытых источниках находится информация о покрывающей способности некоторых маршевых тестов, например MATS, MATS++ и других. На основе этой информации строятся тесты, для проверки правильности работы тестирующих алгоритмов ОЗУ. Тесты предполагают наличие файла неисправностей в формате csv. Общий алгоритм тестирования проверки работоспособности маршевых тестов един и сводится к таблице \ref{sec:testing:march_testing}.
\pagebreak

\begin{longtable}[p]{| >{\raggedright}p{0.3\textwidth}                     
                     | >{\raggedright}p{0.3\textwidth}
                     | >{\raggedright\arraybackslash}p{0.3\textwidth}|}
  \caption{Тестирование поведения системы при наличии в памяти неисправностей}
  \label{sec:testing:march_testing} \tabularnewline

  \hline
      Название тест-кейса и его описание & Ожидаемый результат & Фактический результат результат \\
   \hline
   Работоспособность маршевых тестов\\ 
   1) Инициализировать объект MemorySystem входными параметрами: путь к ini-файлу, путь к файлу конфигурации памяти, объем памяти, путь к файлу неисправностей; \\
   2) запустить цикл работы симулятора.

   &
   1) В консоли отображается информация о запуске системы и объеме симулируемой памяти;\\
   2) после наступления периода обновления появляется сообщение о несовпадении эталонной и тестовой сигнатур;\\
   3) появляется уведомление о запуске маршевого теста;\\
   4) маршевый тест не пройден, обнаружены неисправности.

   &
   Тест пройден \\
   \hline
\end{longtable}


%   \begin{longtable}{| >{\raggedright}p{0.20\linewidth} 
%                   | >{\raggedright}p{0.28\linewidth} 
%                   | >{\raggedright}p{0.28\linewidth} 
%                   | >{\raggedright\arraybackslash}p{0.12\linewidth}|}
%    \caption{Тестовые сценарии} \label{tab:long} \\

%    \hline
%    Идентификатор тест-кейса & Описание тест-кейса & Ожидаемый результат & Тестовый сценарий пройден Да/Нет \\
%    \endfirsthead

% \multicolumn{3}{l}%
% {{\raggedright Продолжение таблицы \thetable{}}} \\
% \hline
%    Идентификатор тест-кейса & Описание тест-кейса & Ожидаемый результат & Тестовый сценарий пройден Да/Нет \\
% \endhead
%    \hline
%    Login-1 &
%    			\vspace{-6.5mm} 
%    			\begin{enumerate} 
%    				\item[1)] Ввести в поле <<Login>> и поле <<Passcode>> не корректные значения;
% 				\item[2)] нажать на кнопку <<Log In>>.
% 			\end{enumerate}
%    			 & Появится сообщение <<Invalid login>> & Да \\

%    \hline
%    Login-2 & 
%    			\vspace{-6.5mm}
%    			\begin{enumerate} 
%    				\item[1)] Ввести в поле корректное значение <<Login>>;
% 				\item[2)] ввести в поле <<Passcode>> значение из символов;
% 				\item[3)] нажать на кнопку <<Log In>>.
% 			\end{enumerate}
% 			& Появится сообщение <<Passcode length should be 4 number>> & Да \\

%    \hline
%    Login-3 & 
%    			\vspace{-6.5mm}
%    			\begin{enumerate} \item[1)] Ввести в поле корректное значение <<Login>>;
% 				\item[2)] ввести в поле <<Passcode>> значение из более чем 4 цифр;
% 				\item[3)] нажать на кнопку <<Log In>>.
% 			\end{enumerate}
%    			& Появится сообщение <<Passcode length should be 4 number>> & Да \\
%    	\hline
%    Login-4 & 
%    			\vspace{-6.5mm}
%    			\begin{enumerate} \item[1)] Ввести в поле корректное значение <<Login>>;
% 				\item[2)] ввести в поле <<Passcode>> корректное значение из 4 цифр;
% 				\item[3)] нажать на кнопку <<Log In>>.
% 			\end{enumerate}
%    			& 
%    			\vspace{-6.5mm}
%    			\begin{enumerate} \item[1)] Закрывается страница авторизации;
% 				\item[2)] появляется главная страница приложения.
% 			\end{enumerate}
% 			& Да \\
%    \hline
%    Login-5 & \vspace{-6.5mm} \begin{enumerate} \item[1)] Ввести в поле корректное значение <<Login>>;
% 				\item[2)] ввести в поле <<Passcode>> значение из 4 цифр не являющуюся паролем данного пользователя;
% 				\item[3)] нажать на кнопку <<Log In>>.
% 			\end{enumerate}
%    			& Появится сообщение <<Invalid Passcode>> & Да \\
%    	\hline
%    	Login-6 & \vspace{-6.5mm} \begin{enumerate} \item[1)] Ввести в поле несколько пробелов, а после корректное значение <<Login>>;
% 				\item[2)] ввести в поле <<Passcode>> корректное значение из 4 цифр;
% 				\item[3)] нажать на кнопку <<Log In>>.
% 			\end{enumerate}
%    			& \vspace{-6.5mm} \begin{enumerate} \item[1)] Закрывается страница авторизации;
% 				\item[2)] появляется главная страница приложения.
% 			\end{enumerate}
% 			& Да \\
%    	\hline
%    	Register-1 & \vspace{-6.5mm} \begin{enumerate} \item[1)] Ввести в поле значение <<Login>>;
% 				\item[2)] ввести в поле <<Passcode>> корректное значение из 4 цифр без пробелов;
% 				\item[3)] ввести в поле <<Confirm Passcode>> другое значение из 4 цифр без пробелов;
% 				\item[4)] нажать на кнопку <<Register>>.
% 			\end{enumerate}
%    			& Появляется сообщение <<Not corrected passcode>>;
% 			& Да \\
%    	\hline
%    	Register-2 & \vspace{-6.5mm} \begin{enumerate} \item[1)] Ввести в поле значение <<Login>>;
% 				\item[2)] ввести в поле <<Passcode>> корректное значение из 4 цифр без пробелов;
% 				\item[3)] ввести в поле <<Confirm Passcode>> схожее значение с полем <<Passcode>>
% 				\item[4)] нажать на кнопку <<Register>>.
% 			\end{enumerate}
%    			& \vspace{-6.5mm} \begin{enumerate} \item[1)] Закрывается страница авторизации;
% 				\item[2)] появляется страница заполнения профиля.
% 			\end{enumerate}
% 			& Да \\
%    	\hline
%    	Register-3 & \vspace{-6.5mm} \begin{enumerate} \item[1)] Ввести в поле значение <<Login>>;
% 				\item[2)] ввести в поле <<Passcode>> значение из более чем 4 цифр без пробелов;
% 				\item[3)] ввести в поле <<Confirm Passcode>> схожее значение с полем <<Passcode>>
% 				\item[4)] нажать на кнопку <<Register>>.
% 			\end{enumerate}
%    			& Появится сообщение <<Passcode length should be 4 number>> & Да \\
%    	\hline
%    	Register-4 & \vspace{-6.5mm} \begin{enumerate} \item[1)] Ввести в поле значение <<Login>> существующего пользователя;
% 				\item[2)] ввести в поле <<Passcode>> корректное значение из 4 цифр без пробелов;
% 				\item[3)] ввести в поле <<Confirm Passcode>> схожее значение с полем <<Passcode>>
% 				\item[4)] нажать на кнопку <<Register>>.
% 			\end{enumerate}
%    			& Появится сообщение <<User consist in system>> & Да \\
%    	\hline
%    	Register-5 & \vspace{-6.5mm} \begin{enumerate} \item[1)] Ввести в поле значение <<Login>>;
% 				\item[2)] ввести в поле <<Passcode>> строку с пробелами и следом 4 цифры;
% 				\item[3)] ввести в поле <<Confirm Passcode>> схожее значение с полем <<Passcode>>
% 				\item[4)] нажать на кнопку <<Register>>.
% 			\end{enumerate}
%    			& Появится сообщение <<Passcode can't contain space>> & Да \\
%    	\hline
%    	Register-6 & \vspace{-6.5mm} \begin{enumerate} \item[1)] Ввести в поле значение <<Login>>;
% 				\item[2)] ввести в поле <<Passcode>> корректное значение из 4 цифр без пробелов;
% 				\item[3)] ввести в поле <<Confirm Passcode>> схожее значение из 4 цифр с пробелов;
% 				\item[4)] нажать на кнопку <<Register>>.
% 			\end{enumerate}
%    			& Появится сообщение <<Passcode can't contain space>> & Да \\
%    	\hline
%    	Logout-1 & \vspace{-6.5mm} \begin{enumerate} \item[1)] Ввести в поле корректное значение <<Login>>;
% 				\item[2)] ввести в поле <<Passcode>> корректное значение из 4 цифр;
% 				\item[3)] нажать на кнопку <<Log In>>;
% 				\item[3)] нажать на кнопку <<Log Out>>.
% 			\end{enumerate}
%    			& Появится страница с окном авторизации & Да \\
%    \hline
%   \end{longtable}