\section{Тестирование программного средства}
\label{sec:test}

Для обеспечения качества программного средства используются модульные и интеграционные тесты. Модульные тесты позволяют проверять отдельные участки программного средства на правильность их выполнения при определенных входных параметрах. Это позволяет достаточно быстро проверить, не привело ли очередное изменение кода к появлению определенного рода ошибок в уже оттестированных местах приложения, а также облегчает обнаружения и устранения таких ошибок. Цель модульного тестирования - изолировать отдельные участки приложения и доказать, что по отдельности эти части являются работоспособными. 

Интеграционные тесты позволяют проверить правильность взаимодействия различных модулей программного средства друг с другом. При интеграционном тестирование отдельные модули приложения объеденяются в специальные модули и тестируются в совокупности.

В таблице 5.1 представлены тестовые сценарии, разработанные в ходе выполнения дипломного проекта для данного мобильного программного средства.
\begin{table}[ht]
  \caption{Работники, занятые в проекте}
  \label{table:econ:programmers}
  \begin{tabular}{| >{\raggedright}m{0.20\textwidth} 
                  | >{\raggedright}m{0.28\textwidth} 
                  | >{\raggedright}m{0.28\textwidth} 
                  | >{\raggedright\arraybackslash}m{0.12\textwidth}|}
   \hline
   Идентификатор тест-кейса & Описание тест-кейса & Ожидаемый результат & Тестовый сценарий пройден Да/Нет \\
   \hline
   Login-1 & Ввести в инпуты значения & Появится сообщение & Да \\
   \hline
  \end{tabular}
\end{table}