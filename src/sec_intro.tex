\sectioncentered*{Введение}
\addcontentsline{toc}{section}{Введение}
\label{sec:intro}

Информационные и коммуникационные технологии развиваются с каждым годом все быстрее, и их применение для нужд лечебно-профилактических учреждений становится все более широким. Тем не менее, уровень коммуникации между медучреждениями, между клиникой и пациентами отстает от всех существующих уже возможностей, которые существуют на сегодняшний день.

Все чаще встает необходимость оперативного, информационно насыщенного и безопасного взаимодействия отдельно взятого лечебного учреждения с другими, а также со своими пациентами. Такое комплексное взаимодействие на регулярной основе создает предпосылки для создания профессиональных медицинских сетей и принципиально новых сервисов для врачей и пациентов.

Электронная медицинская карта пациента – это комплекс данных о состоянии здоровья пациента и назначаемом ему лечении, которые хранятся и обрабатываются в электронном виде.
Электронная медицинская карта позволяет быстро находить существующую и добавлять новую информацию обо всех случаях оканаия пациенту медицинской помощи, а также в автоматизированном режиме формировать медицинские документы.

Однако чаще всего электронная медицинская карта хранится в каждом медучереждении своя, тем самым возникает трудность у пациента при переходе от одного медучреждения в другой. 
В последние годы все больше людей переходят к потребности организации своего времени. Приложения электронная медицинская карта начали приспосабливаться к данному тайм-менеджменту и давать возможность пациентам следить за приемом медицинских препаратов, а также когда проходить анализы или осмотры в медучреждениях. 

В связи с этим появилась перспективная необходимость реализовать мобильный сервис, который бы помогал пользователям-пациентам следить за своим здоровьем, путем отслеживания результатов обследований в различных медучреждениях, планировать прием медицинских препаратов и статистику состояния пациента в течении периода, путем использования различных датчиков мобильного телефона. Дополнительно, необходимо реализовать поддержку приложения на всех современных мобильных платформах.
