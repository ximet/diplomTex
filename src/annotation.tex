\sectioncentered*{Аннотация}
\thispagestyle{empty}

\begin{center}
  \begin{minipage}{0.82\textwidth}
    на дипломный проект <<Мобильное программное средство Органайзер здоровья пациента>> студента УО <<Белорусский государственный университет информатики и радиоэлектроники>> Бардт~Д.\,В.
  \end{minipage}
\end{center}

\emph{Ключевые слова}: мобильное программное средство; органайзер; обработка изображений; экспортирование в PDF; картотека.

\vspace{4\parsep}

Дипломный проект выполнен на 6 листах формата А1 с пояснительной запиской на~\pageref*{LastPage} страницах, без приложений справочного или информационного характера. 
Пояснительная записка включает \total{section}~глав, \totfig{}~рисунков, \tottab{}~таблиц, \toteq{}~формулы, \totref{}~литературный источник.

Целью дипломного проекта является разработка удобного в использовании мобильного приложения, пригодного для повседневного использования, для анализа состояния здоровья пациента.

Для достижения цели дипломного проекта была разработано приложение на платформе Cordova, используя JavaScript библиотеку ReactJS, предназначенное для представления информации о состоянии здоровья пациента.
Приложение может быть использована в реальных проектах, как частичный модуль.

Во введении производится ознакомление с проблемой, решаемой в дипломном проекте.

В первой главе производится обзор предметной области проблемы решаемой в данном дипломном проекте.
Приводятся необходимые теоретические сведения, а также производится обзор существующих разработок.

Во второй главе производится моделирование предметной области.

В третьей главе производится проектирование программного средства, архитектура и методы использованные для реализации ПО в рамках дипломного проекта.

В четвертой главе производится обзор реализованного ПО.
Описываются его составные части и особенности использования.
Приводятся результаты практических испытаний и производится сравнение с существующим ПО.

В пятой главе производится технико"=экономическое обоснование разработки.

В заключении подводятся итоги и делаются выводы по дипломному проекту, а также описывается дальнейший план развития проекта.

\clearpage