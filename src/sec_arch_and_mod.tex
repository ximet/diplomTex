\section{Проектирование программного средства} % (fold)
\label{sec:arch_and_mod}

\subsection{Разработка программного развертывания}
\label{sub:arch_and_mod:graphlib}

Прежде чем приступать к непосредственной реализации программного средства, необходимо определиться с архитектурой коллективной работы с приложением.
Во-первых, необходимо провести анализ необходимой аппаратной конфигурации, на которой будут работать части конечного программного средства, и описать их взаимодействие между собой. Для описания узлов и их связей будем использовать диаграмму развертывания:
\begin{figure}[ht]
\centering
  \includegraphics[scale=0.5]{device.png}  
  \caption{ Диаграмма развертывания }
  \label{fig:domain:manual_structure:credit_device}
\end{figure}
На основе вышеизображенной диаграммы можно сделать следующие выводы:

\begin{enumerate}
  \item узлы могут располагаться в различных частях мира и взаимодействовать между собой через сеть Интернет;
  \item сервер базы данных поддерживаются в рабочем состоянии отдельно от основного сервера;
  \item клиент, осуществляющий работу с системой с помощью HTTP/HTTPS;
  \item HTTPS протокол применяется как к клиентам, так и к всевозможным серверам, для осуществления обмена запросами и ответами на них.
\end{enumerate}

\subsection{Разработка архитектуры БД }
\label{sub:arch_and_mod:graphlib}

Перед тем, как приступать к процессу разработки, сначала нужно смоделировать базу данных, которая является оcновой будущего приложения. Чтобы база данных не вызывала трудностей при работе с ней, когда будет заполнена информацией, крайне желательно, чтобы она соответствовала трем нормальным формам:
\begin{itemize}
  \item защита от CSRF-атак (cross-site request forgery, межсайтовая подделка запроса), при которых данные пользователя могут быть переданы на другой сайт (например, сайт злоумышленника или сайт платежной системы) для совершения некой вредоносной операции.
  \item Первая нормальная форма. Переменная отношения находится в первой нормальной форме тогда и только тогда, когда в любом допустимом значении отношения каждый его кортеж содержит только одно значение для каждого из атрибутов. 
  \item Вторая нормальная форма. Переменная отношения находится во второй нормальной форме тогда и только тогда, когда она находится в первой нормальной форме, и каждый неключевой атрибут функционально полно зависит от ее потенциального ключа.
  \item Третья нормальная форма. Переменная отношения находится в третьей нормальной форме тогда и только тогда, когда она находится во второй нормальной форме, и отсутствуют транзитивные функциональные зависимости неключевых атрибутов от ключевых.
\end{itemize}
В дипломном проекте будет разрабатываться нереляционная база данных. Делаться это будет путем создания JavaScript объектов, которые в последствии дополняться необходимыми полями и впоследствии будут хранится в базе данных также в виде объектов. Преимущества данного подхода являются: 
\begin{enumerate}
  \item решение проблемы масштабируемости;
  \item решение проблемы доступности;
  \item применение различных типов хранилищ;
  \item возможность создания базы данных без задания схемы;
  \item возможность использования многопроцессорности;
  \item сокращение времени разработки;
  \item скорость и производительность.
\end{enumerate}
Схема базы данных разрабатываемого приложения имеет следующий вид:
\begin{figure}[ht]
\centering
  \includegraphics[scale=0.7]{db.png}  
  \caption{ Модель базы данных программного средства }
  \label{fig:domain:manual_structure:credit_db}
\end{figure}

\subsection{Разработка алгоритма ПС }
\label{sub:arch_and_mod:alholib}
На рисунке ~\ref{fig:domain:manual_structure:alho_sp} представлена схема работы клиентского приложения, которая демонстрирует алгоритм работы программного средства в целом. Из схемы можно увидеть, что пользователь может взаимодействовать с двумя частями клиентского приложения:

\begin{figure}[ht]
\centering
  \includegraphics[scale=0.25]{systemRegistAlho.png}  
  \caption{ Схема регистрации приложения }
  \label{fig:domain:manual_structure:alho_regist}
\end{figure}

\begin{figure}[ht]
\centering
  \includegraphics[scale=0.25]{systemEditMed.png}  
  \caption{ Схема редактирования заключения пользователя }
  \label{fig:domain:manual_structure:alho_edit}
\end{figure}

\begin{enumerate}
  \item с модулем карточек, который позволяет взаимодействовать различными способами с карточками;
  \item с модулем работы с изображением.
\end{enumerate}

На рисунке ~\ref{fig:domain:manual_structure:alho_regist} представлена схема работы регистрации приложения. Пользователь должен ввести валидные емэйл и пароль, чтобы пройти регистрацию. Суть алгоритма заключается в проверке введенной пользователем информации и сохранением полученого объекта пользователь в базу данных.

На рисунке ~\ref{fig:domain:manual_structure:alho_edit} представлена схема работы редактирование медицинских заключений. В самом начале алгоритма производится поиск существующего заключения в базе данных и если такое медицинское заключение найдено, то оно заменяется вновь созданным заключением.

Веб-сервер приложения будет представлять собой приложение, построенное на концепции модифицированной архитектуры MVC со слоями:
\begin{enumerate}
  \item слой отражения модели на таблицу в базе данных(ML);
  \item слой доступа к данным (DAL);
  \item слой бизнес логики (BLL);
  \item слой управления (Controller).
\end{enumerate}
\begin{figure}[ht]
\centering
  \includegraphics[scale=0.4]{structurServer.png}  
  \caption{ Структура серверного-приложения при использовании модифицированной архитектуры MVC }
  \label{fig:domain:manual_structure:structural_server}
\end{figure}

Слой управления представляет собой управленческий механизм, который обеспечивает связь между пользователем сервера и непосредственно самого сервера. Его основная задача отвечать на запросы пользователя и выдавать ему корректные результаты. Также одной из дополнительных функций - является контроль переданных данных от пользователя.

Слой доступа к данным предоставляет интерфейс для определенного рода операций взаимодействия внешних слоев с источником данных, в данном случае с объектами моделей.

Слой бизнес-логики описывает основные функции приложения, предназначенные для достижения поставленных перед ним целей.

Слой отраэения модели на таблицу в базе данных представляет собой совокупность объектов, связанных между собой и дополненных в последующем определенного вида параметрами для легкой доступности самих моделей.

На рисунке ~\ref{fig:domain:manual_structure:structural_server} представлена структура серверного-приложения на основе модифицированной MVC архитектуры. Такая структура имеет следующие преимущества:
\begin{itemize}
  \item подход позволяет производить параллельную разработку;
  \item возможность заменять слои, не нарушая тем самым целостность приложения;
  \item возможность протестировать каждый из слоев независимо друг от друга.
\end{itemize}

Клиентское приложение будет представлять собой приложение, построенное на концепции архитектуры Redux с основными принципами:
\begin{enumerate}
  \item состояние всего приложения сохранено в дереве объектов внутри одного хранилища (Store);
  \item единственный способ изменить состояние - это применить действие (Action);
  \item для определения трансформации хранилища используются чистые функции редьюсеры (Reducer).
\end{enumerate}
\begin{figure}[ht]
\centering
  \includegraphics[scale=0.5]{reduxFlow.png}  
  \caption{ Структура клиентского-приложения при использовании архитектуры Redux }
  \label{fig:domain:manual_structure:structural_client}
\end{figure}

Действия - это структура, которая передает данные из приложения в хранилище. Они являются единственными источниками информации для хранилища. Действия яаляются обычными JavaScript объектами.

Редьюсер - чистая функция, которая принимает предыдущее состояние и действие и возвращает следующее состояние.

Хранилище - объект, который соеденяет все части вместе. Хранилище берет на себя следующие задачи:
\begin{itemize}
  \item содержать состояние приложения (state);
  \item предоставлять доступ к состоянию приложения;
  \item предоставлять возможность обновления состояния.
\end{itemize}

На рисунке ~\ref{fig:domain:manual_structure:structural_client} представлена структура клиентского-приложения она имеет следующие преимущества:
\begin{itemize}
  \item позволяет использовать компонентный подход в разработке;
  \item делает изменение состояние предсказуемым;
  \item позволяет использовать функциональный подход.
\end{itemize}
\begin{figure}[ht]
\centering
  \includegraphics[scale=0.25]{spAlho.png}  
  \caption{ Схема работы программы }
  \label{fig:domain:manual_structure:alho_sp}
\end{figure}
\subsection{Разработка алгоритма крипто-сервиса }
\label{sub:arch_and_mod:alholib-crypto}

В дипломном проекте необходимо передавать данные от серверной части к клиентской. Чтобы не произошел перехват данных и их изменения необходимо разработать крипто-сервис, который обеспечит сохранность и достоверность информации, передаваемой от сервера к клиенту.

Для этого был разработан алгоритм AES-128 - симметричный алгоритм блочного шифрования. Этот алгоритм преобразует один 128-битный блок в другой, используя секретный ключ который нужен для такого преобразования. Для расшифровки полученного 128-битного блока используют второе преобразование с тем же секретным ключом.

Размер блока всегда равен 128 бит. Размер ключа также имеет фиксированный размер. Чтобы зашифровать произвольный текст любым паролем можно поступить так: 

\begin{itemize}
  \item получить хеш от пароля;
  \item преобразовать хеш в ключ по правилам описанным в стандарте AES;
  \item разбить текст на блоки по 128 бит;
  \item зашифровать каждый блок функцией cipher.
\end{itemize}
\begin{figure}[ht]
\centering
  \includegraphics[scale=0.5]{alho_aes.png}  
  \caption{ Структура алгоритма AES }
  \label{fig:domain:manual_structure:structural_aes}
\end{figure}
На рисунке ~\ref{fig:domain:manual_structure:structural_aes} представлен алгоритм шифрования на вход которого приходит 128-битный блок данных plaintext и расписание ключей w, которое получается после KeyExpansion. 16-байтый plaintext он записывает в виде матрицы s размера 4*Nb, которая называется состоянием AES, и затем Nr раз применяет к этой матрице 4 преобразования. В конце он записывает матрицу в виде массива и подаёт его на выход — это зашифрованный блок. Каждое из четырёх преобразований очень простое.

AddRoundKey берёт из расписания ключей одну матрицу размера 4Nb и поэлементно добавляет её к матрице состояния. Если два раза применить AddRoundKey, то ничего не изменится, поэтому преобразование обратное к AddRoundKey это оно само.

SubBytes заменяет каждый элемент матрицы состояния соответвующим элементом таблицы SBox: sij = SBox[sij]. Преобразование SubBytes обратимо. Обратное к нему находится с помощью таблицы InvSBox.

