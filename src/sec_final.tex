\sectioncentered*{Заключение}
\addcontentsline{toc}{section}{Заключение}

Итогом выполнения дипломного проекта является мобильное программное средство <<Органайзер здоровья пациента>>.

До реализации данного программного продукта были изучены подходы реализаций аналогов программного средства. Для каждого аналога были выявлены плюсы и минусы и сделаны выводы. Также был произведен анализ платформ и библиотек, для реализации клиентской части. В результате проведенного анализа и на основании требований заданных к программному средству, был определен оптимальный набор платформ и библиотек.

Для серверной части приложения был произведен анализ существующих облачных сервисов, предоставляющих возможность использовать веб-сервера и веб-сервисы. Из всех аналогов был выбран облачный сервер Amazon, который успешно подходил под все требования.

Спроектированы необходимые схемы алгоритмов и схема базы данных. Все поставленные перед разработкой цели выполнены, реализована вся функциональность, представленная в спецификации требований. Произведено тестирование разработанного программного средства с использованием модульных и интеграционных тестов.

При реализации программного продукта были использованы следующие технологии:

\begin{itemize}
  \item платформа Cordova;
  \item NodeJS для сервера приложения;
  \item ReactJS для клиентского приложения;
  \item MongoDB;
  \item ORM Mongoose.
\end{itemize}

Также в ходе разработки дипломного проекта была рассмотрена экономическая сторона и рассчитан экономический эффект от внедрения программного средства и показатели эффективности использования программного средства у пользователя. В результате расчетов подтвердилась экономическая целесообразность разработки. 