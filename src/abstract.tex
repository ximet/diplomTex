\sectioncentered*{Реферат}
\thispagestyle{empty}

\begin{center}
Пояснительная записка 96 c., 18 рис., 11 табл., 42 формулы и 13 источников.\\
\MakeUppercase{мобильное программное обеспечение, органайзер, здоровье, картотека, заключение, медицина}
\end{center}

Объектом исследования является программное обеспечение электронная медицинская карта для отслеживания состояния здоровья пациента и назначаемом ему лечении.

Цель работы -- разработка мобильного программного средства для анализа состояния пациента, результаты которого, получены с помощью введения информации вручную или обработки фотографии.

Разработка данного программного средства обеспечит формирование базы данных по отдельным конструктивным элементам с хранением сведений о результатах исследования и состоянии пользователя. 

Проведен анализ методов обработки изображений, способов декодирова-ния полученных результатов обработки.

Рассмотрены способы создания мобильных приложений, возможность использования аппаратных возможностей телефона.

Разрабатываемое программное обеспечение должно стать элементом системы автоматизации по наблюдению за состоянием здоровья пациента, его анализу и нахождению способов лечения, найденных заболеваний или откло-нений.

\clearpage
